%%This is a very basic article template.
%%There is just one section and two subsections.
\documentclass{scrartcl}
\usepackage[colorlinks, urlcolor=blue]{hyperref}
\usepackage{fullpage}
\usepackage{graphics} 
\usepackage{graphicx}
\usepackage{verbatim}

\author{Jeroen Ooms}
\title{OpenCPU Installation and Administration Manual for Ubuntu 12.04
(LTS)}
\subtitle{Version 0.70-0}


\begin{document}

\maketitle

\section*{About}

\noindent This document describes installation and administration of the \texttt{OpenCPU}
server. \texttt{OpenCPU} runs on \texttt{Ubuntu Linux 12.04 (LTS)}, either on a
\texttt{i386} (32 bit) or \texttt{x64} architecture. \texttt{Ubuntu Server}
edition is recommended, but all versions of \texttt{Ubuntu 12.04}, including
\texttt{Ubuntu Server}, \texttt{Ubuntu Desktop}, \texttt{Kubuntu},
\texttt{EDUbuntu}, etc are supported. \texttt{Ubuntu 12.04} is a Long Term
Support edition (LTS), and the official nickname of this version of
\texttt{Ubuntu} is \emph{Precise Pangolin}, or for short: \emph{Precise}.
Version 12.04 refers to the month in which it was released: April 2012.\\

The installation described in this document should work out of the box on a
clean version of \texttt{Ubuntu} 12.04. We highly recommend taking advantage of
modern technology by using some sort of virtual server or cloud environment to
host \texttt{OpenCPU}. Virtual servers are easy to install, delete or clone, and
furthermore most cloud environments offer convenient preinstalled version of
\texttt{Ubuntu Linux}. This way installation of \texttt{OpenCPU} is easy and can
be done in a couple of minutes. If you have no idea where to start: Amazon EC2
is a nice and affordable pay-by-hour hosting service which includes great
preinstalled versions of \texttt{Ubuntu Linux}.\\

Note that \texttt{OpenCPU} is Open Source software and does not
come with any guarantees or warranties. The project is still in an early stage
and most likely the software will contain bugs. Please post feedback, problems,
comments, suggestions, etc, of any kind on github: \\

\url{https://github.com/jeroenooms/opencpu/issues} \\

Thank you for giving it a try. More information on the project can be found on
the OpenCPU website: \url{http://opencpu.org}.

\newpage 

\section{Installation}

The following instructions will deploy a server with:

\begin{itemize}
  \item \texttt{Ubuntu 12.04}
  \item \texttt{OpenCPU 0.70}
  \item \texttt{R 2.15}
\end{itemize}

\noindent The 12.04 version of \texttt{Ubuntu} ships with the following
versions of third party dependencies:

\begin{itemize}
  \item \texttt{Linux kernel 3.2.0}
  \item \texttt{Apache 2.2.22} (includes \texttt{mod-ssl})
  \item \texttt{Apparmor 2.7.102}
  \item \texttt{Nginx 1.1.19}
\end{itemize}

\subsection{Packages}


This section will show how to install \texttt{OpenCPU} on \texttt{Ubuntu 12.04
(LTS)}. The \texttt{OpenCPU} software has been wrapped up in a number of
packages. These include:
\begin{itemize}
  \item \texttt{opencpu-server} -- \texttt{OpenCPU} API server.
  \item \texttt{opencpu-cache} -- load balancing / caching server
  (\textbf{optional})
  \item \texttt{opencpu-cran} -- installs all CRAN packages (\textbf{optional})
\end{itemize}
Only installation of the \texttt{opencpu-server} package is required. The
\texttt{opencpu-cache} and \texttt{opencpu-cran} packages contain optional
additional functionality. When installing either of these packages, all
dependencies will automatically be installed as well.

\subsection{Installing Ubuntu Linux}

\noindent The current version of \texttt{OpenCPU} requires an \texttt{Ubuntu
12.04} system, either on a \texttt{i386} (32 bit) and \texttt{x64} architecture.
\texttt{Ubuntu Server} edition is recommended, but all versions of
\texttt{Ubuntu 12.04}, including \texttt{Ubuntu Server}, \texttt{Ubuntu
Desktop}, \texttt{KUbuntu}, \texttt{EDUbuntu}, etc are supported. The preferred
way of running \texttt{OpenCPU} is on a clean \texttt{Ubuntu Server} edition. A
free copy of the \texttt{Ubuntu Server} installation disc can be obtained from
the \texttt{Ubuntu} download page: \\

\url{http://www.ubuntu.com/download/server/download} \\

\noindent If you would like to run \texttt{OpenCPU} on an Amazon EC2 server, the best way
is to use one of the official AMI's as provided by the \texttt{Ubuntu} team: \\

\url{http://cloud-images.ubuntu.com/precise/current/} \\

\noindent Another possibility is to install a \texttt{Ubuntu} on a virtual server inside
another OS. For example, on Windows and \texttt{Linux} the free VMware Player or VMware
vSphere Hypervisor (ESXi) can be used to run one or more virtual servers, and on
Mac OSX, one can use Parallels. This way you can install \texttt{Ubuntu} and \texttt{OpenCPU}
safely on top of an existing system, and easily remove it at any time.

\subsection{Getting the system up-to-date}

\noindent Throughout this manual it is assumed that you have shell access,
either through a terminal or over SSH. Before beginning installation of
\texttt{OpenCPU}, make sure you are running \texttt{Ubuntu 12.04 (Precise)} by
entering:

\begin{verbatim}
    cat /etc/*release
\end{verbatim}
If it turns out the system is running an older version of \texttt{Ubuntu}, upgrade the OS
to 12.04 first. If the system is indeed 12.04, continue by updating existing
software packages on the system to the latest versions:

\begin{verbatim}
    sudo apt-get update
    sudo apt-get upgrade
\end{verbatim}
Once the system is up to date, you can begin installing \texttt{OpenCPU}.

\subsection{OpenCPU Installation}

Start by adding the \texttt{opencpu-0.7} package respository to the system:

\begin{verbatim}
    sudo apt-get install python-software-properties
    sudo add-apt-repository ppa:opencpu/opencpu-0.7
\end{verbatim}
The system will ask for confirmation on importing the public key. After the
repository has been added to the system, update the package list:

\begin{verbatim}
    sudo apt-get update
\end{verbatim}
Once this has succeeded \texttt{OpenCPU} can be installed. Enter:

\begin{verbatim}
    sudo apt-get install opencpu-server
\end{verbatim}
This will be sufficient to get started with \texttt{OpenCPU}. \texttt{OpenCPU} has many
dependencies, and first installation might take a while. If the installation
finished without any problems, it will display the ip address of the host at
the very end, which you can can open in your browser and use to test the
server.

\subsection{Uninstall \texttt{OpenCPU}}

To remove \texttt{OpenCPU} from a system, enter:

\begin{verbatim}
    sudo apt-get remove --purge opencpu-server
    sudo apt-get autoremove
\end{verbatim}

Note that if objects were saved in the store, you might have to remove this
manually by removing contents of the \texttt{/mnt/export} directory.

\section{Administration}

The default install of \texttt{OpenCPU} contains 2 services that you can independently
control. To control the \texttt{OpenCPU} web server, use:

\begin{verbatim}
    sudo service opencpu-server {start | stop | restart}
\end{verbatim}

This will completely enable/disable \texttt{OpenCPU} and restart the webserver. To
enable/disable the sandbox, use:

\begin{verbatim}
    sudo service opencpu-sandbox {start | stop | restart}
\end{verbatim}
The sandbox is based on AppArmor and is there to prevent users from running
malicious code through the R API. It is enabled by default, but for debugging or
testing purposes you might want to temporary disable the sandbox.

\subsection{Configuration Files}

All configuration files for \texttt{OpenCPU} are stored in
\texttt{/etc/opencpu}. The main server configuration file is called
\texttt{server.conf}. To edit it, use a text editor, like nano:

\begin{verbatim}
    sudo nano /etc/opencpu/server.conf
\end{verbatim}
The configuration file is formatted in JSON. The syntax should be pretty
intuitive. Make sure that when you edit it, you validate that it is still
valid JSON or the server might not be able to read it. Below a list of the
settings that can be modified in the server.conf file:

\begin{itemize}
   \item \texttt{github.clientid} -- Github client ID for this domain.
   \item \texttt{github.secretfile} -- points to file with the github secret.
   \item \texttt{disable.eval} -- disables evaluation of code in parameters.
   \item \texttt{enable.cors} -- enable/disable CORS headers.
   \item \texttt{whitelist} -- either \texttt{false} or an array of packages
   that are allowed.
   \item \texttt{key.length} -- length of the hashkeys for store objects.
   \item \texttt{job.timeout} -- soft limit on R jobs (in seconds)
   \item \texttt{time.limit} -- harder limit on R jobs (in seconds)
   \item \texttt{max.storesize} -- max size of an object in the store.
   \item \texttt{storedir} -- directory of the object store.
   \item \texttt{cache.store} -- cache-control value for getting objects (in
   seconds)
   \item \texttt{cache.call} -- cache-control header value for function calls
   (in seconds)
   \item \texttt{cache.help} -- cache-control header value for help files (in
   seconds)
   \item \texttt{return.warnings} -- if warnings should be included in case of
   an error.
   \item \texttt{accesslog} -- file with access log entries
   \item \texttt{errorlog} -- file with error log entries
   \item \texttt{syslib} -- library of \texttt{OpenCPU} system dependencies
   \item \texttt{libpaths} -- paths where to look for packages
   \item \texttt{preload} -- an array of packages to preload when the server
   starts.
\end{itemize}
After editing configurations, the server has to be restarted to activate them:
\begin{verbatim}
    sudo service opencpu-server restart
\end{verbatim}

\subsection{Apache2}

\texttt{OpenCPU} is based on the Apache2 webserver. It should not be necessary to
manually edit these configurations, but in case you want to hack around,
below a short intro to managing Apache2 on \texttt{Ubuntu}. To manage the server daemon:

\begin{verbatim}
    sudo service apache2 {start | stop | restart}
\end{verbatim}
This command calls the \texttt{/etc/init.d/apache2} script which should usually
not be edited. The main configuration file for apache2 is located at

\begin{verbatim}
    /etc/apache2/httpd.conf
\end{verbatim}
However by convention this file should also rarely be edited. Custom
configurations are located at:

\begin{verbatim}
    /etc/apache2/mods-available/
    /etc/apache2/sites-available/
\end{verbatim}
These custom configurations can be activated and de-activated as follows:

\begin{verbatim}
    sudo a2enmod mod_R
    sudo a2dismod mod_R
    sudo a2ensite opencpu-user
    sudo a2dissite opencpu-user
\end{verbatim}
These commands create or remove symoblic inside links to available configuration
files inside the following directories:

\begin{verbatim}
    /etc/apache2/mods-enabled/
    /etc/apache2/sites-enabled/
\end{verbatim}
All files in these directories are automatically included by the main
\texttt{httpd.conf} file. The \texttt{OpenCPU} sites are defined in the
following files:

\begin{verbatim}
    /etc/apache2/sites-available/opencpu
    /etc/apache2/sites-available/opencpu-user
    /etc/apache2/sites-available/opencpu-apps
\end{verbatim}

The Apache2 log files \texttt{access.log} and \texttt{error.log} are located at
\begin{verbatim}
    /var/log/apache2/
\end{verbatim}

\subsection{HTTPS access and SSL certificates}

By default Apache only listens to port 80 (HTTP). To enable HTTPS, run the
following commands:

\begin{verbatim}
    sudo a2enmod ssl
    sudo a2ensite default-ssl
    sudo service apache2 restart
\end{verbatim}
This will enable SSL access to your website. Note that SSL is not compatible
with the caching server; HTTPS requests will never be cached on the server.

\subsection{Certificates}
By default, Apache uses self signed a.k.a. snakeoil certificates. This is
convenient for development servers, but in a production setting these should be
replaced by SSL certificates signed by an official Certificate Authority. \\

\noindent The https configurations and locations of the certificates are defined
in

\begin{verbatim}
    /etc/apache2/sites-available/default-ssl
\end{verbatim}
This file also contains detailed comments with configuration instructions. \\

\noindent When the caching server is enabled, the SSL certificates should be
installed on NginX instead of Apache. Modify the file: 

\begin{verbatim}
    /etc/nginx/sites-available/opencpu
\end{verbatim}
and look for the directives \texttt{ssl\_certificate} and
\texttt{ssl\_certificate\_key}.

\section{The Caching / Load balancing Server}

\texttt{OpenCPU} has an optional caching/load balancing server based on NGINX.
To install it, enter:

\begin{verbatim}
    sudo apt-get install opencpu-cache
\end{verbatim}
When enabled, the server will automatically take over port 80 and 443. The
caching server can be enabled/disabled using:

\begin{verbatim}
    sudo service opencpu-cache {start | stop | restart}
\end{verbatim}
The server configuration can be edited to include additional back-ends and hence
work as a load balancer as well. The balancer is configured using NGINX
configurations found in \texttt{/etc/nginx}.

\end{document}
